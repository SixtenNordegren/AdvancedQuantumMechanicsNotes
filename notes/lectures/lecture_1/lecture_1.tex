\documentclass{elsarticle}
\usepackage{amsmath}
\usepackage{hyperref}
\usepackage{enumitem}


\title{Notes on Lecture 1 \\ Advanced Quantum Mechanics}
\author{Sixten Nordegren}
\date{\today}
 
\begin{document}
\maketitle
	\section*{Introduction}
	Everyone is introducing themselves.
	\subsection*{Teachers}
	\begin{itemize}

		\item Igor Pitovski \\
		\textbf{Email}: \url{igor.pikovski@fysik.su.se} 
		\item Elisabeth Edvardsson \\
		\textbf{Email}: \url{elisabeth.edvardsson@fysik.su.se} 
	\end{itemize}
	\subsection*{About the course}
	\begin{itemize}
		\item 15 lectures
		\item 1 lab demonstration (M.Bouremanne) \\ \texttt{Not sure i spelled 
			this right!}
		\item 15 tutorials: exercises in class (Edvardsson, in person)
		\item 4 Hand in homework assignments (max $20 \%$ of final grade)
		\item Book: J.J Sakurai "Modern QM" \\
			Will go through Ch.1-6, most but all topics covered.
			Some complementary topics. Whatever edition is fine.
	\end{itemize}

	\section*{Course content}
	\begin{itemize}
		\item Feeds into many other subjects:
			\begin{itemize}
				\item \textbf{QFT}
				\item \textbf{Standard Model}
				\item \textbf{Particle physics}
				\item \textbf{Condensed matter physics}
				\item \textbf{Quantum gravity}
				\item \textbf{Cosmology}

			\end{itemize}
	\end{itemize}
	\subsection*{Topics to be covered}
	\begin{itemize}
		\item Mathematical foundations
		\begin{itemize}
			\item Description Hilbert space
			\item Measurements
			\item Mathematical tools
		\end{itemize}
	\item Composite systems \& entanglement
	\item Two - level systems / Spin systems
	\item Continuous variable systems
	\item Dynamics , Schrödinger evolution, Heisenberg picture, path integrals
	\end{itemize}
	\begin{itemize}
		\item Symmetries
		\item Angular momentum
		\item Approximate techniques for solutions:
			\begin{itemize}
				\item Perturbation theory (time-dependent)
				\item Scattering theory
			\end{itemize}
	\end{itemize}
	\section*{Next lecture}
	\begin{itemize}
		\item Mathematical foundations, (Linear algebra with Dirac notation)
	\end{itemize}
\end{document}
